
\documentclass[a4paper]{article}
\usepackage[cm]{fullpage}
\usepackage{fancyhdr}
\usepackage[pdfborder={0 0 0}]{hyperref}
\usepackage{verbatim}
\usepackage{titlesec}
\usepackage{changepage}

\setlength{\headheight}{12pt}
\pagestyle{fancy}
\lfoot{Extended CV}
\rfoot{\url{https://lukeblaney.co.uk/cv}}
\cfoot{}
\renewcommand{\headrulewidth}{0.0pt}
\renewcommand{\footrulewidth}{0.0pt}
\titleformat{\subsection}
{\normalsize\bfseries}{\thesubsection}{1em}{}
\titlespacing\subsection{0pt}{0pt}{0pt}
\newenvironment{detail}{\begin{adjustwidth}{10pt}{}}{\end{adjustwidth}}
\begin{document}

\begin{center}\textsc{\LARGE Luke Blaney BSc (Hons)}\end{center}

\begin{tabular}{ l l }

Email Address: & \href{mailto:cv@lukeblaney.co.uk}{cv@lukeblaney.co.uk}\\

\end{tabular}

\section*{Employment}

\begin{itemize}


\item Financial Times: Feburary 2023 - present
\subsection*{Cyber Security Director}
\begin{detail}
Setting strategic direction for cyber security at the FT

\begin{itemize}
	\item Defined the vision and set the strategy for IT Risk and Cyber Security.
	\item Introduced and oversaw a new process for renewing and refreshing technical policies across the organisation.
	\item Responsible for security incident management and contributed to wider crisis management.
	\item Spent 3 months on secondment to the FT's parent company, Nikkei, in Toyko.  Helped with their product security function and building closer ties between the two companies.
\end{itemize}
\end{detail}

\item Financial Times: April 2022 - January 2023
\subsection*{Interim VP of Cyber Security}
\begin{detail}
Leading the company's cyber security function through a period of organisational change.

\begin{itemize}
	\item Built a central cyber security function, combining a number of teams which had previously reported into different parts of the organisation.
	\item Line managed a team of a dozen people from different disciplines, including Engineers, Risk Analysts and those in IT governance roles.
	\item Steered the direction of Security work across the technology department, and the organisation as a whole.
	\item Worked with a third party consultancy to build a strategic roadmap and help prioritise security initiatives.
	\item Involved in recruitment of a permanent VP of Cyber Security.
\end{itemize}
\end{detail}

\item Financial Times: September 2021 - March 2022
\subsection*{Principal Engineer - Observability, Edge Delivery \& Cyber Security}
\begin{detail}
Working on the strategic direction in three key areas across the FT's technology department.

\begin{itemize}
	\item Responsible for line management of 2 in-house engineering teams, plus the relationship with an off-shore 3rd party engineering team.
	\item Worked on vendor management of key suppliers within relevant domains and set the direction of the FT's relationships with them.
	\item Steered changes to the department's approach to tracking security risk with an aim to make it more outcome-focused.
	\item Analysis of how observability and security tooling is being used across the company and putting in place strategic plans for these, including end-of-life roadmaps for tools which give minimal value.
	\item Recruited engineers for vacancies within our teams, including both internal \& external candidates.  Aimed to keep diversity and equity at forefront of hiring decisions, including actively modifying our recruitment process following feedback from neurodiverse candidates.
\end{itemize}
\end{detail}

\item Financial Times: September 2020 - September 2021
\subsection*{Principal Engineer - Cyber Security}
\begin{detail}
Leading the engineering function of the FT's cyber security team.

\begin{itemize}
	\item Acted as tech lead and line manager for engineers in the cyber security team
	\item Managed security incidents across the business; collaborated with Operations team to improve process.
	\item Took responsibility for shaping the team's roadmap and seeing projects through to completion.
	\item Contributed to group-wide strategy, ensuring security was well represented.
	\item Led work on the democritisation of security data, to enable engineering teams to make better decisions around their own risks, beginning with an aggregated view of vulnerablity data.
	\item Advised on security related concerns across all the FT's engineering teams.
\end{itemize}
\end{detail}

\pagebreak

\item Financial Times: February 2018 - August 2020
\subsection*{Principal Engineer - Reliability Engineering}
\begin{detail}
Running a new team, set up to improve reliability of systems and reduce duplication of effort across technology teams at the FT.

\begin{itemize}
	\item Formed a brand new team whose members had backgrounds in a variety of different domains.
	\item Jointly led the team, taking on line management repsonsibilites for half its members.
	\item Supported engineers from other teams joining us on three-month secondments.
	\item Actively involved in the full recruitment process for new members of the team.
	\item Was a department rep for the move of our company headquarters.  In collaboration with reps from across the business, I worked representing the 300+ people in our area, ensuring they could make the move seemlessly.
	\item Oversaw the delivery of several core technical tools, now used extensively across the company, including a monitoring aggregation platform, tech migration tracker and change management system.
	\item Worked with engineering teams throughout the department to build a clearer understanding of our estate (including bespoke software, enterprise systems and SaaS solutions)
	\item Liased with procurement team for all suppliers under our cost centre.  Handled renewals, ad-hoc licence increases and taking new contracts through the full procurement process.
\end{itemize}
\end{detail}


\item Financial Times: October 2016 - February 2018
\subsection*{Architect - Content}
\begin{detail}
Worked on our Universal Publishing Platform, which delivers Content \& Metadata from a range of editorial tools to our websites, apps and third-party B2B clients.

\begin{itemize}
	\item Responsible for architectural decisions across a whole programme of work (6 dev teams, in 2 countries)
	\item Collaborated with Editorial Tech team to design and build a metadata management tool for the newsroom.
	\item Worked closely with stakeholders from Editorial, B2B and Technology departments.
	\item Successfully repositioned our relationship with an existing supplier so they'd take on operational responsibility for the software they were building.
	\item Devised a framework for tracking progress on a multi-year project to decommission a suite of legacy software, in a way that made sense to both delivery managers and engineers.
	\item Took on line management responsilities for integration engineers in our team.
\end{itemize}
I think my proudest accomplishment here was transforming a team heavily reliant on having an architect to one where engineers felt empowered to make their own architectural decisons.
\end{detail}

\item Financial Times: December 2015 - October 2016
\subsection*{Platform Architect - Operational Intelligence}
\begin{detail}
Technical lead on a team responsible for observability tools used across the department (covering monitoring, metrics \& log aggregation).  We also introduced a new system for keeping track of the FT's technical estate, which provided automated runbook creation and monitoring management.  My role included:
\begin{itemize}
	\item System design, development and backlog prioritisation
	\item Upskilling a team from disparate backgrounds and recruitment of junior developers
	\item Collaborating with delivery to move the team to a kanban agile workflow
	\item Working with teams across the FT and third-party suppliers
\end{itemize}
\end{detail}

\item Financial Times: January 2015 - November 2015
\subsection*{Integration Engineer - Strategic Products}
\begin{detail}
Led the migration of the FT's mobile apps to cloud-based infrastructure (primarily AWS) to facilitate the switch-off of physical kit in our datacentres.  This involved:
\begin{itemize}
	\item Adpating existing workloads to run on cloud
	\item Recruitment of a contractor to assist with the project
	\item Architecting \& development of a cloud-native zero-downtime deployment pipeline
\end{itemize}
\end{detail}

\small{\texttt{[Note: Officially, my job title did not change for this period. Included separately for clarity.]}}


\pagebreak

\item Financial Times: December 2011 - December 2014
\subsection*{Labs Developer - FT Labs}
\begin{detail}
In this role, I gained experience with the full stack of web development technologies, from using bleeding-edge browser features in the FT's HTML5 Web App, through to server mangement and configuration using tools like Puppet, Varnish and Apache.
\begin{itemize}
	\item Worked on the original incarnation of FastFT, a realtime news service.  This included a customer-facing frontend, an API and bespoke CMS for editorial users.
	\item Led FT Labs' internal tooling workstream, which focused on infrastructure improvements and developer experience (including deployment, live error collection and monitoring aggregation).  My role involved engineering, architectural decisions and co-ordinating the work of others in this area.
	\item Following a cyber attack in 2013, I architected, helped engineer and cordinated roll-out of a tool to put multifactor authentication in front of all of the FT's staff-facing tools.
\end{itemize}
\end{detail}


\item Assanka: November 2010 - December 2011
\subsection*{Web Developer}
\begin{detail}
Worked on various web projects for a range of clients, doing both backend and frontend work. Backends included Wordpress, an in-house framework and vanilla PHP. Frontend javascript varied from simple jQuery-based interfaces, to large object-orientated systems which utilised many HTML5 features.
Assanka was acquired by the Financial Times at the end of 2011 and was later rebranded ``FT Labs".
\end{detail}

\end{itemize}

\section*{Talks \& Panels}
\begin{itemize}

\item{\bf Panel: Building security into your engineering workflow | LeadDev 2021}\\
Moderated an online panel about improving cyber security across engineering organisations.
\item{\bf Panel: Observability Strategies for Distributed Systems | InfoQ Live 2020}\\
Discussed approches for observability when using microservices at this one-day online conference.
\item{\bf Monitoring All the Things: Keeping Track of a Mixed Estate | QCon London 2020; Continuous Lifecycle 2020}\\
Spoke about effectively monitoring the tech estate of a company which uses a variety of different technologies and monitoring approaches.  Was also on a panel about the value of microservices at the QCon event.
\item{\bf Panel: Standarisation and autonomy in our tech choices | Engine Room, London 2019}\\
Discussed the value of consistency versus local optimisation in our technical approaches at the FT's internal tech conference.
\item{\bf Panel: Do we only measure things that are easy to measure? | Engine Room, London 2017}\\
Chaired a panel discussing the use of metrics in tech and product decisions at the FT's internal tech conference.
\item{\bf Creating a multi-factor authentication solution | Varnish Summit \& Awards, Los Angeles 2016}\\
Spoke about how the FT used varnish to protect our estate with MFA and tackle phishing attacks.  Also collected the Varnish Innovation Award at the same event.
\item{\bf Panel: Page Load Performance | Edgeconf, London 2014 }\\
Discussed performance best practices when using cutting edge web technologies.
\item{\bf One VCL to rule all our environments | Varnish User Group, Berlin 2013}\\
Spoke about the consolidation of our caching and routing logic across environments using puppet.
\end{itemize}
For a full list of Talks \& Panels I've done, see \url{https://lukeblaney.co.uk/talks/}


\section*{Education}
\begin{itemize}

\item Edinburgh University: 2006 - 2010
\subsection*{BSc (Hons) Computer Science and Physics}
\begin{detail}
A joint degree, with an emphasis on Computer Science, covering specialist areas such as Distributed Systems, Multi-Agent Semantic Web Systems, Computer Security and Communication \& Networking.
My Honours project used Semantic Web technologies to provide data for a Semi-Automatic Guesstimation system.
\subsubsection*{Publications}
\em A Single-Significant-Digit Calculus for Semi-Automated Guesstimation \em\\
Jonathan Abourbih, {\bf Luke Blaney}, Alan Bundy and Fiona McNeill\\
IJCAR 2010
\end{detail}

\item Lagan College, Belfast: 1999 - 2006
\begin{itemize}\item {\bf A-levels}: Maths (A), Irish (A), Physics (B), Chemistry (C)
\item {\bf GCSEs}: Maths (A*), Science (A*A*), Irish (A*) + 1 A, 4 Bs and 1 C.
\end{itemize}

\end{itemize}

\section*{Previous Work Experience}
\begin{itemize}

\item Edinburgh University Students Association, Entertainments Crew - {\bf Lighting Technician}: 2007 - 2010
\item BBC Irish Language department - {\bf One week's Work Placement}: Summer 2009
\item J Sainsbury plc - {\bf Customer Services Assistant}: May 2005 - Dec 2005
\item Andor Technology Ltd. - {\bf One week's Work Placement}:  February 2005
\item Queen's University Belfast Media Services - {\bf One week's Work Placement}:  June 2003.

\end{itemize}

\section*{Positions of responsibility}

\begin{itemize}
\item London Revolution Cheer - {\bf Secretary}: 2013 - 2014.
\item PHP London (including PHP UK conference 2012) - {\bf Treasurer}: 2011 - 2012
\item Edinburgh University Theatre Company - {\bf Webmaster}: 2009 - 2010
\item Edinburgh University Cheerleading Society - {\bf Publicity Manager/Webmaster}: 2008 - 2010
\item Edinburgh University Juggling Society (including 2009 Scottish Juggling convention)
	\begin{itemize}
	\item {\bf Secretary}: 2007 - 2008
	\item {\bf President}: 2008 - 2009
	\item {\bf Treasurer}: 2009 - 2010
	\end{itemize}
\item Edinburgh University Footlights - {\bf Webmaster}: 2009 - 2010
\end{itemize}

\end{document}
