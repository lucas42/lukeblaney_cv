
\documentclass[a4paper]{article}
\usepackage[cm]{fullpage}
\usepackage{fancyhdr}
\usepackage[pdfborder={0 0 0}]{hyperref}
\usepackage{verbatim}
\usepackage{titlesec}
\usepackage{changepage}
\usepackage[left=1.5cm,right=1.5cm,top=2cm,bottom=2cm]{geometry}

\setlength{\headheight}{12pt}
\pagestyle{fancy}
\rfoot{\url{https://lukeblaney.co.uk/cv}}
\cfoot{}
\renewcommand{\headrulewidth}{0.0pt}
\renewcommand{\footrulewidth}{0.0pt}
\titleformat{\subsection}
{\normalsize\bfseries}{\thesubsection}{1em}{}
\titlespacing\subsection{0pt}{0pt}{0pt}
\newenvironment{detail}{\begin{adjustwidth}{10pt}{}}{\end{adjustwidth}}
\begin{document}

\begin{center}\textsc{\LARGE Luke Blaney BSc (Hons)}\end{center}

\begin{tabular}{ l l }

Email Address: & \href{mailto:cv@lukeblaney.co.uk}{cv@lukeblaney.co.uk}\\

\end{tabular}


\begin{detail}
Senior technical leader with expertise guiding 3-6 engineering squads through complex technical challenges.  Experienced at architecting and delivering resilient systems, tailored to the needs of its customers, including non-functional requirements.  Adept at stakeholder management, in particular bridging the gap between technical and non-technical staff.
\end{detail}

\section*{Key Experience}

\begin{itemize}
	\item Providing technical leadership to engineering squads (up to 3 concurrent squads when also doing line management; or up to 6 squads in an architecture role)
	\item Introducing a data-driven approach to areas which had previously been done in a piecemeal fashion across the company, including monitoring, vulnerablity management and migration tracking.
	\item Working with a full range of stakeholders, both technical and non-technical, from junior members of staff through to the members of the Management Board.
	\item Architecting robust, resilient systems to solve complex technical challenges.
	\item Managing technical incidents and complicated cross-departmental human/technology problems.
\end{itemize}
\section*{Employment}

\begin{itemize}


\item Financial Times: February 2023 - March 2025
\subsection*{Cyber Security Director}
\begin{detail}
Set strategic direction for cyber security at the FT.

\begin{itemize}
	\item Defined the vision and set the technical strategy across a critical domain.
	\item Senior stakeholder management - acting as a bridge between business colleagues and engineers.
	\item Introduced and oversaw a new process for renewing and refreshing technical policies across the organisation, increasing discoverability of policies and improving adherence.
	\item Led security incident management and contributed to a new company-wide crisis management team.
	\item Spent 3 months on secondment to the FT's parent company, Nikkei, in Tokyo.  Assisted their product security function and strengthened ties between the two companies.
\end{itemize}
\end{detail}

\item Financial Times: April 2022 - January 2023
\subsection*{Interim VP of Cyber Security}
\begin{detail}
Led the company's cyber security function through a period of organisational change.

\begin{itemize}
	\item Built a new centralised cyber security function, consolidating people, process and budgets from across the organization.  This reduced barriers in communication and led to better-aligned security outcomes.
	\item Line managed a mixed-discipline team of a dozen people, including Engineers, Risk Analysts and IT governance roles.
	\item Steered the direction of Security work across the technology department, and the organisation as a whole.
	\item Worked with a third party consultancy to build a strategic roadmap and help prioritise security initiatives.
	\item Involved in recruitment of a permanent VP of Cyber Security.
\end{itemize}
\end{detail}

\item Financial Times: February 2018 - March 2022
\subsection*{Principal Engineer - Reliability Engineering, Cyber Security, Observability \& Edge Delivery}
\begin{detail}
Ran a selection of engineering teams throughout this period, including building a new one from scratch.  Led the company-wide strategic direction in these areas, as well as managing the delivery of initiatives within the relevant teams.

\begin{itemize}
	\item Technical leadership for 3 squads of engineers
	\item Responsible for line management of 2 in-house engineering teams and the relationship with a 3rd off-shore team.
	\item Successfully led and completed the company-wide migration of several engineering tools from on-premise software to Software-as-a-Service alternatives, including Code Hosting, Issue Tracking and Single Sign-On platforms.  As part of this, I did the vendor management, led technical migration strategy and project managed the end-to-end moves.
	\item Recruited engineers for vacancies within our teams, including both internal \& external candidates.  Aimed to keep diversity and equity at forefront of hiring decisions, including actively modifying our recruitment process following feedback from neurodiverse candidates.
	\item Was a department rep for the move of our company headquarters.  In collaboration with reps from across the business, I worked representing the 300+ people in our area, ensuring they could make the move seamlessly.
	\item Led the strategy and implementation of democratisation of security data — enabling engineering teams to make better decisions around their own risks.
	\item Rolled out a new company-wide security incident management process and took the lead on managing security incidents.
\end{itemize}
\end{detail}

\item Financial Times: October 2016 - February 2018
\subsection*{Architect - Content}
\begin{detail}
Worked on our Universal Publishing Platform, which delivers Content \& Metadata from a range of editorial tools to our websites, apps and third-party B2B clients.

\begin{itemize}
	\item Responsible for architectural decisions across 6 engineering squads working on a platform of 100+ microservices.
	\item Designed a suite of new APIs to power an in-house metadata management tool for the newsroom.
	\item Planned and co-ordinated a complex multi-team multi-year piece of work to replace a suite of legacy on-premise business-critical software with a mix of SaaS solutions and cloud-hosted bespoke microservices.
	\item Pivoted the 'architect' role from a decision bottleneck to a more consultative role, giving engineering squads more autonomy for day-to-day decisions.
	\item Worked closely with stakeholders from Editorial, B2B and Technology departments.
	\item Led a migration to Kubernetes, simplifying operations and reducing technical risk across 5 engineering squads and a critical area for the business.
	\item Successfully repositioned our relationship with an existing supplier giving them operational responsibility for the software they were building.
	\item Line management responsibilities for 2 engineers.
\end{itemize}
\end{detail}

\item Financial Times: December 2015 - October 2016
\subsection*{Platform Architect - Operational Intelligence}
\begin{detail}
Technical lead for a new team responsible for observability tools used across the company.  We replaced the organisation's CMDB with a tool which better bridged the gap between Engineering and Operations teams.  My role included:
\begin{itemize}
	\item System design, development and backlog prioritisation
	\item Upskilling a team from disparate backgrounds and recruitment of junior developers
	\item Moving the team's processes to a kanban agile delivery workflow
	\item Stakeholder and supplier management
\end{itemize}
\end{detail}

\item Financial Times: January 2015 - November 2015
\subsection*{Integration Engineer - Mobile Apps}
\begin{detail}
Led the migration of the FT's mobile apps from on-premise to cloud-based infrastructure in AWS.
\begin{itemize}
	\item Adapted existing workloads to run on cloud
	\item Recruited a contractor to assist with the migration
	\item Architected \& developed a cloud-native zero-downtime deployment pipeline
\end{itemize}
\end{detail}

\end{itemize}

\section*{Earlier Career}
\begin{itemize}

\item FT Labs - {\bf Labs Developer}: December 2011 - December 2014
\item Assanka - {\bf Web Developer}: November 2010 - December 2011

\end{itemize}

\pagebreak

\section*{Talks \& Panels}
\begin{itemize}

\item Panel Moderator: {\bf Building security into your engineering workflow} | LeadDev 2021
\item Panelist: {\bf Observability Strategies for Distributed Systems} | InfoQ Live 2020
\item Speaker: {\bf Monitoring All the Things: Keeping Track of a Mixed Estate}\\ | QCon London 2020; Continuous Lifecycle 2020
\item Panelist: {\bf Microservices - Are they still worth it?} | QCon London 2020
\end{itemize}
For a full list of Talks \& Panels I've done, see \url{https://lukeblaney.co.uk/talks/}

\section*{Education}
\begin{itemize}

\item Edinburgh University: 2006 - 2010
\subsection*{BSc (Hons) Computer Science and Physics}
\begin{detail}
A joint degree, with an emphasis on Computer Science, covering specialist areas such as Distributed Systems, Multi-Agent Semantic Web Systems, Computer Security and Communication \& Networking.
\subsubsection*{Publications}
\em A Single-Significant-Digit Calculus for Semi-Automated Guesstimation \em\\
Jonathan Abourbih, {\bf Luke Blaney}, Alan Bundy and Fiona McNeill\\
IJCAR 2010
\end{detail}


\end{itemize}

\end{document}
