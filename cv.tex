
\documentclass[a4paper]{article}
\usepackage[cm]{fullpage}
\usepackage{fancyhdr}
\usepackage[pdfborder={0 0 0}]{hyperref}

\setlength{\headheight}{12pt}
\pagestyle{fancy}
\rfoot{\url{http://lukeblaney.co.uk/cv}}
\cfoot{}
\renewcommand{\headrulewidth}{0.0pt}
\renewcommand{\footrulewidth}{0.0pt}  
\begin{document}

\begin{center}\textsc{\LARGE Luke Blaney BSc (Hons)}\end{center}

\begin{tabular}{ l l }

Email Address: & \href{mailto:cv@lukeblaney.co.uk}{cv@lukeblaney.co.uk}\\

\end{tabular}

\section*{Employment}


\begin{itemize}

\item Financial Times: December 2011 - present
\begin{itemize}\item
 {\bf FT Labs - Developer}\\
Following the FT's acquisition of Assanka, it was relaunched as FT Labs.  Whilst my official title is ``Labs developer", I've taken up various roles within the team:
	\begin{itemize}
		\item {\bf Web developer} - I continued working as a web developer on high profile projects such as the FT Web APP and Fast FT (A realtime streaming news platform).  I've also worked on several internal projects too, from simple bash scripts to monitoring systems aimed at aggregating status information from a variety of sources.
		\item {\bf Workstream Lead} - I lead a team of 3 people in change of infrasture and internal tooling for FT Labs.  This involves balancing various demands: support of current infrastructure, creation of infrastructure for new projects as well as the migration to use centrally managed infrastrucure.
		\item {\bf Technical Architect} - I'm frequently involved in planning the technical architecture for projects, both within FT Labs and for the FT as a whole.
		\item {\bf Integration Engineer} - I'm responsible for getting projects up and running in production and also live debugging when things go wrong.
	\end{itemize}
Normally these 4 roles would be done by different people, but as we have quite a small team, I've ended up doing them all concurrently.
\end{itemize}

\item Assanka: November 2010 - December 2011
\begin{itemize}\item
 {\bf Web Developer}\\
This involved working on websites for a range of clients, mostly in finance and the media.  Projects included the Financial Times' blogs platform (run on Wordpress), FT Tilt (whose interface included faceted search), and the FT Web App (acclaimed for its use of HTML5 to provide the feel of a native app)  Mainly working in PHP, javascript, HTML5 and CSS
\end{itemize}

\item Edinburgh University Students Association: 2007 - 2010
\begin{itemize}\item
{\bf  Entertainments Crew - Lighting Technician} \\
This consisted of rigging lighting, programming lighting desks and operating them.  I worked as part of a team with people from a range of disciplines (for example, sound technicians and DJs).  My role involved problem solving, teamwork, knowledge of health and safety and working under pressure.
\end{itemize}


\end{itemize}

\subsection*{Previous Work Experience}
\begin{itemize}

\item BBC Irish Language department - {\bf One week's Work Placement}: Summer 2009
\item J Sainsbury plc - {\bf Customer Services Assistant}: May 2005 - Dec 2005
\item Andor Technology Ltd. - {\bf One week's Work Placement}:  February 2005
\item Queen's University Belfast Media Services - {\bf One week's Work Placement}:  June 2003.


\end{itemize}

\section*{Education}
\begin{itemize}

\item Edinburgh University: 2006 - 2010 \begin{itemize}\item {\bf BSc (Hons) Computer Science and Physics}\\
My joint degree, with an emphasis on Computer Science, covered specialist areas such as Distributed Systems, Multi-Agent Semantic Web Systems, Computer Security and Communication \& Networking.

One of the aspects of my degree which I particularly enjoyed was the opportunity to work with others to solve problems and challenges, for example, pair programming or as part of a larger group.  One group project involved developing a football-playing LEGO\textregistered robot which competed in a tournament against other robots.  The project required teamwork, communication and time management.  

As part of my honours project I created a web interface for a Semi-Automatic Guesstimation system which provided answers to questions asked by the user.   I also enabled the system to get information live from the Semantic Web using SPARQL, a Resource Description Framework (RDF) query language.
\subsection*{Publications}
\em A Single-Significant-Digit Calculus for Semi-Automated Guesstimation \em\\
Jonathan Abourbih, {\bf Luke Blaney}, Alan Bundy and Fiona McNeill\\
Accepted for IJCAR 2010
\end{itemize}

\item Lagan College, Belfast: 1999 - 2006
\begin{itemize}\item {\bf A-levels}: Maths (A), Irish (A), Physics (B), Chemistry (C)
\item {\bf GCSEs}: Maths (A*), Science (A*A*), Irish (A*) + 1 A, 4 Bs and 1 C.

\end{itemize}

\end{itemize}



\section*{Positions of responsibility}


\begin{itemize}
\item {\bf London Revolution Cheer}: 
London Revolution was an amateur cheerleading team.  As secretary I created a small members-only site to help the team keep track of important information.

\begin{itemize}
\item Secretary 2013 - 2014.
\end{itemize}

\end{itemize}

\begin{itemize}
\item {\bf PHP London}: 
PHP London is a monthly user group focusing on PHP which is also responsible for PHP UK - an annual conference aimed at PHP developers.  The conference is growing every year, and in 2012 it was increased to a two-day event for the first time.  As treasurer I was involved in the running of both the user group and also the conference.

\begin{itemize}
\item Treasurer 2011 - 2012.
\end{itemize}

\item {\bf Edinburgh University Theatre Company}: 

As a member of the Theatre Company for two years I was involved in many (mainly technical) roles.  Throughout my involvement I worked in small teams to put on productions; this involved communication, problem solving, and ability to work under pressure.

In my final year, I was responsible for the company's website.  I redesigned the website to hold a much larger amount of data, interface with the ticket system to allow online ticket sales, and moved theatre administration online removing the need for paper forms.   I developed the site using a domain driven design approach, and created an ontology specifically for theatre using the Web Ontology Language (OWL).

\begin{itemize}

\item Webmaster 2009 - 2010.

\end{itemize}

\item {\bf Edinburgh University Cheerleading Society}: 

I was a member of the Cheerleading Society for the four years I was at university.  For the final two, I held the committee position of Publicity Manager.  In this role I moved the emphasis from traditional marketing to social networking and online marketing. I also redesigned the website to make it more accessible, whilst mantaining its upbeat nature.
\begin{itemize}
\item Publicity Manager 2008 - 2010.
\end{itemize}

\item {\bf Edinburgh University Juggling Society}: 

I was a member of the Juggling Society throughout my time at university.  I held elected positions on the society's committee for three years and gained experience in managing budgets, financial planning, organising events, recruiting new members, and leading a small organisation.  Whilst I was president, I led the team which organised and successfully hosted the 2009 Scottish Juggling Convention bringing together people from across the EU for a three day event.
\begin{itemize}
\item Secretary 2007 - 2008.
\item President 2008 - 2009.
\item Treasurer 2009 - 2010
\end{itemize}

\item {\bf Edinburgh University Footlights}: 
I was involved with the Edinburgh University Footlights, a musical theatre company which puts on an annual show, in my last two years at University.  I created a website for their 2010 production, Anything Goes, which including a ticket sales system.  I worked with the production team for the show to define the requirements for the site and establish a project plan to ensure development was completed on schedule and in time for the show to be promoted.  This gave me invaluable experience in managing a high profile project, meeting deadlines, and working in a team.

\begin{itemize}
\item Webmaster 2009 - 2010.
\end{itemize}
\end{itemize}

\end{document}
