
\documentclass[a4paper]{article}
\usepackage[cm]{fullpage}
\usepackage{fancyhdr}
\usepackage[pdfborder={0 0 0}]{hyperref}
\usepackage{verbatim}
\usepackage[compact]{titlesec}

\setlength{\headheight}{12pt}
\pagestyle{fancy}
\rfoot{\url{http://lukeblaney.co.uk/cv}}
\cfoot{}
\renewcommand{\headrulewidth}{0.0pt}
\renewcommand{\footrulewidth}{0.0pt}
\titleformat{\subsection}
{\normalsize\bfseries}{\thesubsection}{1em}{}
\begin{document}

\begin{center}\textsc{\LARGE Luke Blaney BSc (Hons)}\end{center}

\begin{tabular}{ l l }

Email Address: & \href{mailto:cv@lukeblaney.co.uk}{cv@lukeblaney.co.uk}\\

\end{tabular}

\section*{Employment}


\begin{itemize}

\item Financial Times: February 2018 - Present
\subsection*{Principal Engineer - Relability Engineering}
Setting up and leading a new team working to improve reliability of systems and reduce duplication of effort across technology teams at the FT.


\item Financial Times: October 2016 - February 2018
\subsection*{Lead Architect - Content}
Working on our Universal Publishing Platform, delivering Content \& Metadata from a range of editorial tools to our websites, apps and third-party B2B clients. In this role, I was responsible for architectural decisions across a whole programme of work (6 development teams working in 2 countries). I worked closely with product owners, editorial stakeholders and developers in other teams who interact with our platform.\par

From January 2017 onwards, I took on line management responsibilities.


\item Financial Times: December 2015 - October 2016
\subsection*{Platform Architect - Operational Intelligence}
Technical lead on a team responsible for observability tools used across the department (covering monitoring, metrics \& log aggregation).  We also introduced a new system for keeping track of the FT's technical estate, which provided automated runbook creation and monitoring management.  My role included:
\begin{itemize}
	\item System design, development and backlog prioritisation
	\item Upskilling a team from disparate backgrounds
	\item Collaborating with delivery to move the team to a kanban agile workflow
	\item Recruitment of junior developers
	\item Working with teams across the FT and third-party suppliers
\end{itemize}

\item Financial Times: January 2015 - November 2015
\subsection*{Integration Engineer - Strategic Products}
Led the migration of the FT's mobile apps to cloud-based infrastructure (primarily AWS) to facilitate the switch-off of physical kit in our datacentres.  This involved:
\begin{itemize}
\item Adpating existing workloads to run on cloud
\item Recruitment of a contractor to assist with the project
\item Architecting \& development of a cloud-native zero-downtime deployment pipeline
\end{itemize}

\small{\texttt{[Note: Officially, my job title did not change for this period. Included separately for clarity.]}}

\item Financial Times: December 2011 - December 2014
\subsection*{Labs Developer - FT Labs}
In this role, I gained experience with the full stack of web development technologies, from using bleeding-edge browser features in the FT's HTML5 Web App, through to server mangement and configuration using tools like Puppet, Varnish and Apache.

\subsubsection*{Key projects}
\begin{itemize}
\item Worked on the original incarnation of FastFT, a realtime news service.  This included a customer-facing frontend, an API and bespoke CMS for editorial users.

\item Led FT Labs' internal tooling workstream, which focused on infrastructure improvements and developer experience (including deployment, live error collection and monitoring aggregation).  My role involved engineering, architectural decisions and co-ordinating the work of others in this area.

\item Following a cyber attack in 2013, I architected, helped engineer and cordinated roll-out of a tool to put multifactor authentication in front of all of the FT's staff-facing tools.
\end{itemize}


\item Assanka: November 2010 - December 2011
\subsection*{Web Developer}
Worked on various web projects for a range of clients, doing both backend and frontend work. Backends included Wordpress, an in-house framework and vanilla PHP. Frontend javascript varied from simple jQuery-based interfaces, to large object-orientated systems which utilised many HTML5 features.
Assanka was acquired by the Financial Times at the end of 2011 and was later rebranded ``FT Labs".

\end{itemize}

\section*{Talks \& Panels}
\begin{itemize}

\item{\bf Monitoring All the Things: Keeping Track of a Mixed Estate | QCon London 2020}\\
Spoke about how we effectively monitor our tech estate in a company which uses a variety of different technologies and monitoring approaches.
\item{\bf Panel: Standarisation and autonomy in our tech choices | Engine Room (internal FT tech conference), London 2019}
Discussed the value of consistency versus local optimisation in our technical approaches at the FT.
\item{\bf Panel: Do we only measure things that are easy to measure? | Engine Room (internal FT tech conference), London 2017}\\
Chaired a panel discussing the use of metrics in tech and product decisions at the FT.
\item{\bf How the Financial Times created a multi-factor authenication solution | Varnish Summit \& Award Ceremony, Los Angeles 2016}
Spoke about how we used varnish to protect our estate with MFA and tackle phishing attacks.  Also collected the Varnish Innovation Award at the same event.
\item{\bf Panel: Page Load Performance | Edgeconf, London 2014 }\\
Discussed performance best practices when using cutting edge web technologies.
\item{\bf Varnish @ Financial Times: One VCL to rule all our environments | Varnish User Group, Berlin 2013}\\
Spoke about the consolidation of our caching and routing logic across environments using puppet.
\end{itemize}

\section*{Education}
\begin{itemize}

\item Edinburgh University: 2006 - 2010 \subsection*{BSc (Hons) Computer Science and Physics}
A joint degree, with an emphasis on Computer Science, covering specialist areas such as Distributed Systems, Multi-Agent Semantic Web Systems, Computer Security and Communication \& Networking.
My Honours project used Semantic Web technologies to provide data for a Semi-Automatic Guesstimation system.
\subsubsection*{Publications}
\em A Single-Significant-Digit Calculus for Semi-Automated Guesstimation \em\\
Jonathan Abourbih, {\bf Luke Blaney}, Alan Bundy and Fiona McNeill\\
IJCAR 2010

\item Lagan College, Belfast: 1999 - 2006
\begin{itemize}\item {\bf A-levels}: Maths (A), Irish (A), Physics (B), Chemistry (C)
\item {\bf GCSEs}: Maths (A*), Science (A*A*), Irish (A*) + 1 A, 4 Bs and 1 C.
\end{itemize}

\end{itemize}

\subsection*{Previous Work Experience}
\begin{itemize}

\item Edinburgh University Students Association, Entertainments Crew - {\bf Lighting Technician}: 2007 - 2010
\item BBC Irish Language department - {\bf One week's Work Placement}: Summer 2009
\item J Sainsbury plc - {\bf Customer Services Assistant}: May 2005 - Dec 2005
\item Andor Technology Ltd. - {\bf One week's Work Placement}:  February 2005
\item Queen's University Belfast Media Services - {\bf One week's Work Placement}:  June 2003.

\end{itemize}

\begin{comment}
\section*{Positions of responsibility}

\begin{itemize}
\item London Revolution Cheer - {\bf Secretary}: 2013 - 2014.
\item PHP London (including PHP UK conference 2012) - {\bf Treasurer}: 2011 - 2012
\item Edinburgh University Theatre Company - {\bf Webmaster}: 2009 - 2010
\item Edinburgh University Cheerleading Society - {\bf Publicity Manager/Webmaster}: 2008 - 2010
\item Edinburgh University Juggling Society (including 2009 Scottish Juggling convention)
	\begin{itemize}
	\item {\bf Secretary}: 2007 - 2008 
	\item {\bf President}: 2008 - 2009 
	\item {\bf Treasurer}: 2009 - 2010
	\end{itemize}
\item Edinburgh University Footlights - {\bf Webmaster}: 2009 - 2010
\end{itemize}
\end{comment}

\end{document}
