
\documentclass[a4paper]{article}
\usepackage[cm]{fullpage}
\usepackage{fancyhdr}
\usepackage[pdfborder={0 0 0}]{hyperref}

\setlength{\headheight}{12pt}
\pagestyle{fancy}
\rfoot{\url{http://lukeblaney.co.uk/cv}}
\cfoot{}
\renewcommand{\headrulewidth}{0.0pt}
\renewcommand{\footrulewidth}{0.0pt}  
\begin{document}

\begin{center}\textsc{\LARGE Luke Blaney BSc (Hons)}\end{center}

\begin{tabular}{ l l }

Email Address: & \href{mailto:cv@lukeblaney.co.uk}{cv@lukeblaney.co.uk}\\

\end{tabular}

\section*{Employment}


\begin{itemize}

\item Financial Times: February 2018 - Present
\begin{itemize}\item
 {\bf Principal Engineer - Relability Engineering}\\
Setting up and leading a new team working to improve reliability of systems and reduce duplication of effort across technology teams at the FT.
\end{itemize}

\item Financial Times: October 2016 - February 2018
\begin{itemize}\item
 {\bf Lead Architect - Content}\\
Working on our Universal Publishing Platform, delivering Content \& Metadata from a range of editorial tools to our websites, apps and third-party B2B clients. In this role, I was responsible for architectural decisions across a whole programme of work (6 development teams working in 2 countries). I worked closely with product owners, editorial stakeholders and developers in other teams who interact with our platform.\par

From January 2017 onwards, I took on line management responsibilities.
\end{itemize}

\item Financial Times: December 2015 - October 2016
\begin{itemize}\item
 {\bf Platform Architect - Operational Intelligence}\\
I was technical lead on a team looking after monitoring, metrics \& log aggregation tools. I architected a new API-driven system for keeping track of our systems with automated runbook creation and monitoring management.  My role was a mixture of system design and development - much of my development work focused on the front-end to cover a skills gap within the team.  I was also heavily involved in the recruitment of new developers.
\end{itemize}

\item Financial Times: January 2015 - November 2015
\begin{itemize}

\item {\bf Integration Engineer - Strategic Products} \\
Worked on the migration of various systems to cloud-based infrastructure (primarily AWS) to facilitate the switch-off of physical kit in our datacentres.  I led the migration of all the infrastructure used by the FT's mobile web app to AWS, which involved lots of Puppet code and the recruitment of a contractor to assist in the project. As part of this I architected and built a deployment mechanism using varnish, nodejs and puppetdb, which allowed for zero-downtime promotion of different versions of code running in parallel on groups of EC2 nodes. 

\small{\texttt{[Note: Officially, my job title did not change for this period. Included separately for clarity.]}}
\end{itemize}

\item Financial Times: December 2011 - December 2014
\begin{itemize}
\item {\bf Labs Developer - FT Labs}\\
I began this role focused on web development on a number of projects, such as the FT's HTML5 Web App and FastFT (a realtime news CMS \& front end). I frequently debugged a variety of live issues, from small intermittent bugs which can't be reproduced elsewhere to large show-stopping errors which light up monitoring displays like christmas trees.\par

However, during the role I became more involved in our integrations with backend systems. I wrote a collection of libraries for developers to interact with various pieces of infrastructure (eg Databases, Varnish, Xapian search). I also rearchitectued our Varnish configuration and gave a talk about it at the ``VUG 8" conference in Berlin.\par

The departure of all our team's sysadmins within a short period of time pushed me to become focused on managing the infrastructure itself.  I wrote a considerable amount of puppet configuration and  templates for configuration of other systems (Apache, Varnish etc) I also contributed to all major decisions regarding infrastructure within them team: combining operational requirements, architectural concerns, and the needs of developers interacting with the infrastructure.\par

Towards the end of my time with FT Labs, I took the lead in developing the team's internal tooling.  This involved a combination of hands-on coding and the co-ordination of others working on these tools.  The tools were eclectic in nature including those which dealt with deployments, live error collection and monitoring aggregation.
\end{itemize}

\item Assanka: November 2010 - December 2011
\begin{itemize}
\item {\bf Web Developer}\\
Worked on various web projects for a range of clients, doing both backend and frontend work. Backends included Wordpress, an in-house framework and vanilla PHP. Frontend javascript varied from simple jQuery-based interfaces, to large object-orientated systems which utilised many HTML5 features.
Assanka was acquired by the Financial Times at the end of 2011 and was later rebranded ``FT Labs".
\end{itemize}

\end{itemize}

\subsection*{Previous Work Experience}
\begin{itemize}

\item Edinburgh University Students Association, Entertainments Crew - {\bf Lighting Technician}: 2007 - 2010
\item BBC Irish Language department - {\bf One week's Work Placement}: Summer 2009
\item J Sainsbury plc - {\bf Customer Services Assistant}: May 2005 - Dec 2005
\item Andor Technology Ltd. - {\bf One week's Work Placement}:  February 2005
\item Queen's University Belfast Media Services - {\bf One week's Work Placement}:  June 2003.


\end{itemize}

\section*{Education}
\begin{itemize}

\item Edinburgh University: 2006 - 2010 \begin{itemize}\item {\bf BSc (Hons) Computer Science and Physics}\\
My joint degree, with an emphasis on Computer Science, covered specialist areas such as Distributed Systems, Multi-Agent Semantic Web Systems, Computer Security and Communication \& Networking.

As part of my honours project I created a web interface for a Semi-Automatic Guesstimation system which provided answers to questions asked by the user.   I also enabled the system to get information live from the Semantic Web using SPARQL, a Resource Description Framework (RDF) query language.
\subsection*{Publications}
\em A Single-Significant-Digit Calculus for Semi-Automated Guesstimation \em\\
Jonathan Abourbih, {\bf Luke Blaney}, Alan Bundy and Fiona McNeill\\
Accepted for IJCAR 2010
\end{itemize}

\item Lagan College, Belfast: 1999 - 2006
\begin{itemize}\item {\bf A-levels}: Maths (A), Irish (A), Physics (B), Chemistry (C)
\item {\bf GCSEs}: Maths (A*), Science (A*A*), Irish (A*) + 1 A, 4 Bs and 1 C.
\end{itemize}

\end{itemize}

\section*{Talks}
\begin{itemize}

\item{\bf Monitoring All the Things: Keeping Track of a Mixed Estate | QCon London 2020}\\
Spoke at QCon London in 2020 about monitoring a mixed technical estate across an entire company.
\end{itemize}

\section*{Positions of responsibility}

\begin{itemize}
\item London Revolution Cheer - {\bf Secretary}: 2013 - 2014.
\item PHP London (including PHP UK conference 2012) - {\bf Treasurer}: 2011 - 2012
\item Edinburgh University Theatre Company - {\bf Webmaster}: 2009 - 2010
\item Edinburgh University Cheerleading Society - {\bf Publicity Manager/Webmaster}: 2008 - 2010
\item Edinburgh University Juggling Society (including 2009 Scottish Juggling convention)
	\begin{itemize}
	\item {\bf Secretary}: 2007 - 2008 
	\item {\bf President}: 2008 - 2009 
	\item {\bf Treasurer}: 2009 - 2010
	\end{itemize}
\item Edinburgh University Footlights - {\bf Webmaster}: 2009 - 2010
\end{itemize}

\end{document}
